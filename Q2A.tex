\section{Value at Risk (VaR) Analysis}
\label{sec:q2_value_at_risk_analysis}
\subsection{VaR calculations}
\subsubsection{(i) Constant Model}

The only model that can be formed using just unconditional moments is the constant model $$R_t = \epsilon_t, \epsilon_t \overset{\text{w.n.}}{\sim} N(\mu, \sigma^2),$$
with $\hat \mu = \bar R_t, \hat{\sigma^2} = Var(R_t)$, for $t \in [1,T]$. 
Even for a random walk model both the standard deviation of an observation from the preceding one and a last observation would be needed. Both are not given.


To calculate the VaR of each stock, the historical mean and standard deviation (sd) will be estimated and the respective z value of a standard distribution will be scaled by the sd and shifted by the mean. 
Since the likelihood of the z value can be read off from the normal distribution, the z value corresponding to the likelihood of interest can be chosen.


The scaled and shifted z value corresponds to a maximal log return under the given probability. The typically small log returns of stocks are proportional to percent changes in the stock value. Tue to this, maximal the monetary value of currency that is at risk can be calculated by multiplying the result of the previous calculation with the value of bought stock.
\paragraph{Visa} Caclulations.\newline \indent 





Calculation for 90\% Confidence and $T+1$:

\indent\indent $\text{Quantile}_\text{norm}(0.1) = -1.28,\hat{\mu_{T+1|T}} = 0.0004, \hat{\sigma_{T+1|T}} = 0.0176$

\indent\indent $VaR(\log \text{returns})_{10\%, T + 1} = 0.0004 + 0.0176\times-1.28 = -0.0221$

\indent\indent $VaR_{10\%, T + 1} = \text{ Value in Euros } \times VaR(\log returns) = 10 000 000 \text{\euro} \times-0.0221 = -220709.87 \text{\euro}$\newline




Calculation for 95\% Confidence and $T+1$:

\indent\indent $\text{Quantile}_\text{norm}(0.05) = -1.64,\hat{\mu_{T+1|T}} = 0.0004, \hat{\sigma_{T+1|T}} = 0.0176$

\indent\indent $VaR(\log \text{returns})_{5\%, T + 1} = 0.0004 + 0.0176\times-1.64 = -0.0284$

\indent\indent $VaR_{5\%, T + 1} = \text{ Value in Euros } \times VaR(\log returns) = 10 000 000 \text{\euro} \times-0.0284 = -284484.36 \text{\euro}$\newline




Calculation for 99\% Confidence and $T+1$:

\indent\indent $\text{Quantile}_\text{norm}(0.01) = -2.33,\hat{\mu_{T+1|T}} = 0.0004, \hat{\sigma_{T+1|T}} = 0.0176$

\indent\indent $VaR(\log \text{returns})_{1\%, T + 1} = 0.0004 + 0.0176\times-2.33 = -0.0404$

\indent\indent $VaR_{1\%, T + 1} = \text{ Value in Euros } \times VaR(\log returns) = 10 000 000 \text{\euro} \times-0.0404 = -404114.71 \text{\euro}$\newline


Following the modeling assumptions of this subtask, the available stock data and the given sum invested in the Visa stock, the loss incurring over one day will be smaller than -220709.87$\text{\euro}$, -284484.36$\text{\euro}$  and -404114.71$\text{\euro}$  with a confidence level of 90\%, 95\%  and 99\%.

The due to simplicity of the model, the VaR further into the future is the same as the 1-day VaR.



\paragraph{Microsoft} Caclulations.\newline \indent 




Calculation for 90\% Confidence and $T+1$:

\indent\indent $\text{Quantile}_\text{norm}(0.1) = -1.28,\hat{\mu_{T+1|T}} = 0.0009, \hat{\sigma_{T+1|T}} = 0.0192$

\indent\indent $VaR(\log \text{returns})_{10\%, T + 1} = 0.0009 + 0.0192\times-1.28 = -0.0236$

\indent\indent $VaR_{10\%, T + 1} = \text{ Value in Euros } \times VaR(\log returns) = 10 000 000 \text{\euro} \times-0.0236 = -236303.82 \text{\euro}$\newline




Calculation for 95\% Confidence and $T+1$:

\indent\indent $\text{Quantile}_\text{norm}(0.05) = -1.64,\hat{\mu_{T+1|T}} = 0.0009, \hat{\sigma_{T+1|T}} = 0.0192$

\indent\indent $VaR(\log \text{returns})_{5\%, T + 1} = 0.0009 + 0.0192\times-1.64 = -0.0306$

\indent\indent $VaR_{5\%, T + 1} = \text{ Value in Euros } \times VaR(\log returns) = 10 000 000 \text{\euro} \times-0.0306 = -305890.67 \text{\euro}$\newline




Calculation for 99\% Confidence and $T+1$:

\indent\indent $\text{Quantile}_\text{norm}(0.01) = -2.33,\hat{\mu_{T+1|T}} = 0.0009, \hat{\sigma_{T+1|T}} = 0.0192$

\indent\indent $VaR(\log \text{returns})_{1\%, T + 1} = 0.0009 + 0.0192\times-2.33 = -0.0436$

\indent\indent $VaR_{1\%, T + 1} = \text{ Value in Euros } \times VaR(\log returns) = 10 000 000 \text{\euro} \times-0.0436 = -436424.02 \text{\euro}$\newline


Following the modeling assumptions of this subtask, the available stock data and the given sum invested in the Microsoft stock, the loss incurring over one day will be smaller than -236303.82$\text{\euro}$, -305890.67$\text{\euro}$  and -436424.02$\text{\euro}$  with a confidence level of 90\%, 95\%  and 99\%.

The due to simplicity of the model, the VaR further into the future is the same as the 1-day VaR.


\subsubsection{(ii) AR(0)-GARCH(1,1)}
 The following GARCH model will be fitted 
$$\begin{cases} R_t = \mu_t + \epsilon_t, \epsilon_t = \sigma_t  * z_t, z_t \overset{i.i.d.}{\sim} N(0,1) \\ \sigma^2_t = \alpha_0 + \alpha_1 \epsilon^2_{t-1} + \beta_1 \sigma^2_{t-1} \\ \mu_t = \mu_0 \end{cases}$$

The model fitted in this subtask is more sophisticated than a simple model using only unconditional constants. 
The model will be specified and fitted to the historical data. The fitted model can then be used to make predictions about the stock returns in the future. The predicted mean and sd of the returns can be used to scale and shift a z value like in the previous subtask.
The use of this scaled and shifted z value in the further calculation of the VaR is identical to the previous, current and the later subtasks.
Only the one- and five-days ahead VaR will be reported here, the two- to four-days ahead VaR can be seen in \ref{sec:var2to4days}.
\paragraph{Visa} Caclulations.\newline \indent 




Calculation for 90\% Confidence and $T+1$:

\indent\indent $\text{Quantile}_\text{norm}(0.1) = -1.28,\hat{\mu_{T+1|T}} = 0.0006, \hat{\sigma_{T+1|T}} = 0.0094$

\indent\indent $VaR(\log \text{returns})_{10\%, T + 1} = 0.0006 + 0.0094\times-1.28 = -0.0114$

\indent\indent $VaR_{10\%, T + 1} = \text{ Value in Euros } \times VaR(\log returns) = 10 000 000 \text{\euro} \times-0.0114 = -114043.89 \text{\euro}$\newline




Calculation for 95\% Confidence and $T+1$:

\indent\indent $\text{Quantile}_\text{norm}(0.05) = -1.64,\hat{\mu_{T+1|T}} = 0.0006, \hat{\sigma_{T+1|T}} = 0.0094$

\indent\indent $VaR(\log \text{returns})_{5\%, T + 1} = 0.0006 + 0.0094\times-1.64 = -0.0148$

\indent\indent $VaR_{5\%, T + 1} = \text{ Value in Euros } \times VaR(\log returns) = 10 000 000 \text{\euro} \times-0.0148 = -148128.33 \text{\euro}$\newline




Calculation for 99\% Confidence and $T+1$:

\indent\indent $\text{Quantile}_\text{norm}(0.01) = -2.33,\hat{\mu_{T+1|T}} = 0.0006, \hat{\sigma_{T+1|T}} = 0.0094$

\indent\indent $VaR(\log \text{returns})_{1\%, T + 1} = 0.0006 + 0.0094\times-2.33 = -0.0212$

\indent\indent $VaR_{1\%, T + 1} = \text{ Value in Euros } \times VaR(\log returns) = 10 000 000 \text{\euro} \times-0.0212 = -212065.07 \text{\euro}$\newline




Calculation for 90\% Confidence and $T+5$:

\indent\indent $\text{Quantile}_\text{norm}(0.1) = -1.28,\hat{\mu_{T+5|T}} = 0.0006, \hat{\sigma_{T+5|T}} = 0.0101$

\indent\indent $VaR(\log \text{returns})_{10\%, T + 5} = 0.0006 + 0.0101\times-1.28 = -0.0123$

\indent\indent $VaR_{10\%, T + 5} = \text{ Value in Euros } \times VaR(\log returns) = 10 000 000 \text{\euro} \times-0.0123 = -122764.61 \text{\euro}$\newline




Calculation for 95\% Confidence and $T+5$:

\indent\indent $\text{Quantile}_\text{norm}(0.05) = -1.64,\hat{\mu_{T+5|T}} = 0.0006, \hat{\sigma_{T+5|T}} = 0.0101$

\indent\indent $VaR(\log \text{returns})_{5\%, T + 5} = 0.0006 + 0.0101\times-1.64 = -0.0159$

\indent\indent $VaR_{5\%, T + 5} = \text{ Value in Euros } \times VaR(\log returns) = 10 000 000 \text{\euro} \times-0.0159 = -159321.26 \text{\euro}$\newline




Calculation for 99\% Confidence and $T+5$:

\indent\indent $\text{Quantile}_\text{norm}(0.01) = -2.33,\hat{\mu_{T+5|T}} = 0.0006, \hat{\sigma_{T+5|T}} = 0.0101$

\indent\indent $VaR(\log \text{returns})_{1\%, T + 5} = 0.0006 + 0.0101\times-2.33 = -0.0228$

\indent\indent $VaR_{1\%, T + 5} = \text{ Value in Euros } \times VaR(\log returns) = 10 000 000 \text{\euro} \times-0.0228 = -227895.44 \text{\euro}$\newline


Following the modeling assumptions of this subtask, the available stock data and the given sum invested in the Visa stock, the loss incurring over one day will be smaller than -114043.89$\text{\euro}$, -148128.33$\text{\euro}$  and -212065.07$\text{\euro}$  with a confidence level of 90\%, 95\%  and 99\%.

The loss after 5 days will be smaller than NA$\text{\euro}$, NA$\text{\euro}$  and NA$\text{\euro}$  with a confidence level of 90\%, 95\%  and 99\%.


\paragraph{Microsoft} Caclulations.\newline \indent 




Calculation for 90\% Confidence and $T+1$:

\indent\indent $\text{Quantile}_\text{norm}(0.1) = -1.28,\hat{\mu_{T+1|T}} = 0.0013, \hat{\sigma_{T+1|T}} = 0.013$

\indent\indent $VaR(\log \text{returns})_{10\%, T + 1} = 0.0013 + 0.013\times-1.28 = -0.0154$

\indent\indent $VaR_{10\%, T + 1} = \text{ Value in Euros } \times VaR(\log returns) = 10 000 000 \text{\euro} \times-0.0154 = -154418.3 \text{\euro}$\newline




Calculation for 95\% Confidence and $T+1$:

\indent\indent $\text{Quantile}_\text{norm}(0.05) = -1.64,\hat{\mu_{T+1|T}} = 0.0013, \hat{\sigma_{T+1|T}} = 0.013$

\indent\indent $VaR(\log \text{returns})_{5\%, T + 1} = 0.0013 + 0.013\times-1.64 = -0.0202$

\indent\indent $VaR_{5\%, T + 1} = \text{ Value in Euros } \times VaR(\log returns) = 10 000 000 \text{\euro} \times-0.0202 = -201792.7 \text{\euro}$\newline




Calculation for 99\% Confidence and $T+1$:

\indent\indent $\text{Quantile}_\text{norm}(0.01) = -2.33,\hat{\mu_{T+1|T}} = 0.0013, \hat{\sigma_{T+1|T}} = 0.013$

\indent\indent $VaR(\log \text{returns})_{1\%, T + 1} = 0.0013 + 0.013\times-2.33 = -0.0291$

\indent\indent $VaR_{1\%, T + 1} = \text{ Value in Euros } \times VaR(\log returns) = 10 000 000 \text{\euro} \times-0.0291 = -290659.18 \text{\euro}$\newline




Calculation for 90\% Confidence and $T+5$:

\indent\indent $\text{Quantile}_\text{norm}(0.1) = -1.28,\hat{\mu_{T+5|T}} = 0.0013, \hat{\sigma_{T+5|T}} = 0.0136$

\indent\indent $VaR(\log \text{returns})_{10\%, T + 5} = 0.0013 + 0.0136\times-1.28 = -0.0161$

\indent\indent $VaR_{10\%, T + 5} = \text{ Value in Euros } \times VaR(\log returns) = 10 000 000 \text{\euro} \times-0.0161 = -161413.74 \text{\euro}$\newline




Calculation for 95\% Confidence and $T+5$:

\indent\indent $\text{Quantile}_\text{norm}(0.05) = -1.64,\hat{\mu_{T+5|T}} = 0.0013, \hat{\sigma_{T+5|T}} = 0.0136$

\indent\indent $VaR(\log \text{returns})_{5\%, T + 5} = 0.0013 + 0.0136\times-1.64 = -0.0211$

\indent\indent $VaR_{5\%, T + 5} = \text{ Value in Euros } \times VaR(\log returns) = 10 000 000 \text{\euro} \times-0.0211 = -210771.25 \text{\euro}$\newline




Calculation for 99\% Confidence and $T+5$:

\indent\indent $\text{Quantile}_\text{norm}(0.01) = -2.33,\hat{\mu_{T+5|T}} = 0.0013, \hat{\sigma_{T+5|T}} = 0.0136$

\indent\indent $VaR(\log \text{returns})_{1\%, T + 5} = 0.0013 + 0.0136\times-2.33 = -0.0303$

\indent\indent $VaR_{1\%, T + 5} = \text{ Value in Euros } \times VaR(\log returns) = 10 000 000 \text{\euro} \times-0.0303 = -303357.72 \text{\euro}$\newline


Following the modeling assumptions of this subtask, the available stock data and the given sum invested in the Microsoft stock, the loss incurring over one day will be smaller than -154418.30$\text{\euro}$, -201792.70$\text{\euro}$  and -290659.18$\text{\euro}$  with a confidence level of 90\%, 95\%  and 99\%.

The loss after 5 days will be smaller than NA$\text{\euro}$, NA$\text{\euro}$  and NA$\text{\euro}$  with a confidence level of 90\%, 95\%  and 99\%.

Based on this model, the VaR is smaller for investments into the Visa than for investiments into the Microsoft.

The all VaR obtained by this model are smaller than the VaR obtained by the simplistic model in the first subtask. That is due to the small time window, for which the time dependent variance and the VaR is computed. In this case, the time dependent variance is usually smaller than the historic variance.

\subsubsection{(iii) AR(1)-GARCH(1,1)}
The following GARCH model will be fitted in this subtask
$$\begin{cases} R_t = \mu_t + \epsilon_t, \epsilon_t = \sigma_t  * z_t, z_t \overset{i.i.d.}{\sim} N(0,1) \\ \sigma^2_t = \alpha_0 + \alpha_1 \epsilon^2_{t-1} + \beta_1 \sigma^2_{t-1} \\ \mu_t = \mu_0 + \phi_i R_{t-1} \end{cases}$$ 


here $\mu_t$ is not only modeled as a constant but as a function of the past $\mu_{t-1}$
\paragraph{Visa} Caclulations.\newline \indent 




Calculation for 90\% Confidence and $T+1$:

\indent\indent $\text{Quantile}_\text{norm}(0.1) = -1.28,\hat{\mu_{T+1|T}} = 0.0005, \hat{\sigma_{T+1|T}} = 0.0094$

\indent\indent $VaR(\log \text{returns})_{10\%, T + 1} = 0.0005 + 0.0094\times-1.28 = -0.0115$

\indent\indent $VaR_{10\%, T + 1} = \text{ Value in Euros } \times VaR(\log returns) = 10 000 000 \text{\euro} \times-0.0115 = -114877.12 \text{\euro}$\newline




Calculation for 95\% Confidence and $T+1$:

\indent\indent $\text{Quantile}_\text{norm}(0.05) = -1.64,\hat{\mu_{T+1|T}} = 0.0005, \hat{\sigma_{T+1|T}} = 0.0094$

\indent\indent $VaR(\log \text{returns})_{5\%, T + 1} = 0.0005 + 0.0094\times-1.64 = -0.0149$

\indent\indent $VaR_{5\%, T + 1} = \text{ Value in Euros } \times VaR(\log returns) = 10 000 000 \text{\euro} \times-0.0149 = -148974.66 \text{\euro}$\newline




Calculation for 99\% Confidence and $T+1$:

\indent\indent $\text{Quantile}_\text{norm}(0.01) = -2.33,\hat{\mu_{T+1|T}} = 0.0005, \hat{\sigma_{T+1|T}} = 0.0094$

\indent\indent $VaR(\log \text{returns})_{1\%, T + 1} = 0.0005 + 0.0094\times-2.33 = -0.0213$

\indent\indent $VaR_{1\%, T + 1} = \text{ Value in Euros } \times VaR(\log returns) = 10 000 000 \text{\euro} \times-0.0213 = -212935.98 \text{\euro}$\newline




Calculation for 90\% Confidence and $T+5$:

\indent\indent $\text{Quantile}_\text{norm}(0.1) = -1.28,\hat{\mu_{T+5|T}} = 0.0006, \hat{\sigma_{T+5|T}} = 0.0101$

\indent\indent $VaR(\log \text{returns})_{10\%, T + 5} = 0.0006 + 0.0101\times-1.28 = -0.0123$

\indent\indent $VaR_{10\%, T + 5} = \text{ Value in Euros } \times VaR(\log returns) = 10 000 000 \text{\euro} \times-0.0123 = -122596.95 \text{\euro}$\newline




Calculation for 95\% Confidence and $T+5$:

\indent\indent $\text{Quantile}_\text{norm}(0.05) = -1.64,\hat{\mu_{T+5|T}} = 0.0006, \hat{\sigma_{T+5|T}} = 0.0101$

\indent\indent $VaR(\log \text{returns})_{5\%, T + 5} = 0.0006 + 0.0101\times-1.64 = -0.0159$

\indent\indent $VaR_{5\%, T + 5} = \text{ Value in Euros } \times VaR(\log returns) = 10 000 000 \text{\euro} \times-0.0159 = -159133.89 \text{\euro}$\newline




Calculation for 99\% Confidence and $T+5$:

\indent\indent $\text{Quantile}_\text{norm}(0.01) = -2.33,\hat{\mu_{T+5|T}} = 0.0006, \hat{\sigma_{T+5|T}} = 0.0101$

\indent\indent $VaR(\log \text{returns})_{1\%, T + 5} = 0.0006 + 0.0101\times-2.33 = -0.0228$

\indent\indent $VaR_{1\%, T + 5} = \text{ Value in Euros } \times VaR(\log returns) = 10 000 000 \text{\euro} \times-0.0228 = -227671.12 \text{\euro}$\newline


Following the modeling assumptions of this subtask, the available stock data and the given sum invested in the Visa stock, the loss incurring over one day will be smaller than -114877.12$\text{\euro}$, -148974.66$\text{\euro}$  and -212935.98$\text{\euro}$  with a confidence level of 90\%, 95\%  and 99\%.

The loss after 5 days will be smaller than NA$\text{\euro}$, NA$\text{\euro}$  and NA$\text{\euro}$  with a confidence level of 90\%, 95\%  and 99\%.


\paragraph{Microsoft} Caclulations.\newline \indent 




Calculation for 90\% Confidence and $T+1$:

\indent\indent $\text{Quantile}_\text{norm}(0.1) = -1.28,\hat{\mu_{T+1|T}} = 0.0011, \hat{\sigma_{T+1|T}} = 0.0133$

\indent\indent $VaR(\log \text{returns})_{10\%, T + 1} = 0.0011 + 0.0133\times-1.28 = -0.0159$

\indent\indent $VaR_{10\%, T + 1} = \text{ Value in Euros } \times VaR(\log returns) = 10 000 000 \text{\euro} \times-0.0159 = -159257.48 \text{\euro}$\newline




Calculation for 95\% Confidence and $T+1$:

\indent\indent $\text{Quantile}_\text{norm}(0.05) = -1.64,\hat{\mu_{T+1|T}} = 0.0011, \hat{\sigma_{T+1|T}} = 0.0133$

\indent\indent $VaR(\log \text{returns})_{5\%, T + 1} = 0.0011 + 0.0133\times-1.64 = -0.0207$

\indent\indent $VaR_{5\%, T + 1} = \text{ Value in Euros } \times VaR(\log returns) = 10 000 000 \text{\euro} \times-0.0207 = -207448.05 \text{\euro}$\newline




Calculation for 99\% Confidence and $T+1$:

\indent\indent $\text{Quantile}_\text{norm}(0.01) = -2.33,\hat{\mu_{T+1|T}} = 0.0011, \hat{\sigma_{T+1|T}} = 0.0133$

\indent\indent $VaR(\log \text{returns})_{1\%, T + 1} = 0.0011 + 0.0133\times-2.33 = -0.0298$

\indent\indent $VaR_{1\%, T + 1} = \text{ Value in Euros } \times VaR(\log returns) = 10 000 000 \text{\euro} \times-0.0298 = -297845.55 \text{\euro}$\newline




Calculation for 90\% Confidence and $T+5$:

\indent\indent $\text{Quantile}_\text{norm}(0.1) = -1.28,\hat{\mu_{T+5|T}} = 0.0013, \hat{\sigma_{T+5|T}} = 0.0138$

\indent\indent $VaR(\log \text{returns})_{10\%, T + 5} = 0.0013 + 0.0138\times-1.28 = -0.0164$

\indent\indent $VaR_{10\%, T + 5} = \text{ Value in Euros } \times VaR(\log returns) = 10 000 000 \text{\euro} \times-0.0164 = -163731.38 \text{\euro}$\newline




Calculation for 95\% Confidence and $T+5$:

\indent\indent $\text{Quantile}_\text{norm}(0.05) = -1.64,\hat{\mu_{T+5|T}} = 0.0013, \hat{\sigma_{T+5|T}} = 0.0138$

\indent\indent $VaR(\log \text{returns})_{5\%, T + 5} = 0.0013 + 0.0138\times-1.64 = -0.0214$

\indent\indent $VaR_{5\%, T + 5} = \text{ Value in Euros } \times VaR(\log returns) = 10 000 000 \text{\euro} \times-0.0214 = -213758.49 \text{\euro}$\newline




Calculation for 99\% Confidence and $T+5$:

\indent\indent $\text{Quantile}_\text{norm}(0.01) = -2.33,\hat{\mu_{T+5|T}} = 0.0013, \hat{\sigma_{T+5|T}} = 0.0138$

\indent\indent $VaR(\log \text{returns})_{1\%, T + 5} = 0.0013 + 0.0138\times-2.33 = -0.0308$

\indent\indent $VaR_{1\%, T + 5} = \text{ Value in Euros } \times VaR(\log returns) = 10 000 000 \text{\euro} \times-0.0308 = -307601.04 \text{\euro}$\newline


Following the modeling assumptions of this subtask, the available stock data and the given sum invested in the Microsoft stock, the loss incurring over one day will be smaller than -154418.30$\text{\euro}$, -201792.70$\text{\euro}$  and -290659.18$\text{\euro}$  with a confidence level of 90\%, 95\%  and 99\%.

The loss after 5 days will be smaller than NA$\text{\euro}$, NA$\text{\euro}$  and NA$\text{\euro}$  with a confidence level of 90\%, 95\%  and 99\%.

The estimated VaR don't really differ between the previous model which modeled the mean as constant, and the current one, which models the mean of the time series using an AR(1). 
The reason for this indifference could be, that the conditional mean of stock returns cannot really be predicted from preceding returns. It is already a good aproximation to model this conditional mean as constant. This is consitent with financial theory.

\subsubsection{(iv) AR(1)-GARCH(1,1) with t-Student innovations}
In this subtask, innovations of $R_t$ will not be modeled after a Normal distribution but a t-Student distibution (std) 

\paragraph{Visa} Caclulations.\newline \indent 




Calculation for 90\% Confidence and $T+1$:

\indent\indent $\text{Quantile}_\text{std(5)}(0.1) = -1.46,\hat{\mu_{T+1|T}} = 0.0009, \hat{\sigma_{T+1|T}} = 0.0096$

\indent\indent $VaR(\log \text{returns})_{10\%, T + 1} = 0.0009 + 0.0096\times-1.46 = -0.0131$

\indent\indent $VaR_{10\%, T + 1} = \text{ Value in Euros } \times VaR(\log returns) = 10 000 000 \text{\euro} \times-0.0131 = -130610.68 \text{\euro}$\newline




Calculation for 95\% Confidence and $T+1$:

\indent\indent $\text{Quantile}_\text{std(5)}(0.05) = -1.99,\hat{\mu_{T+1|T}} = 0.0009, \hat{\sigma_{T+1|T}} = 0.0096$

\indent\indent $VaR(\log \text{returns})_{5\%, T + 1} = 0.0009 + 0.0096\times-1.99 = -0.0181$

\indent\indent $VaR_{5\%, T + 1} = \text{ Value in Euros } \times VaR(\log returns) = 10 000 000 \text{\euro} \times-0.0181 = -180885.55 \text{\euro}$\newline




Calculation for 99\% Confidence and $T+1$:

\indent\indent $\text{Quantile}_\text{std(5)}(0.01) = -3.27,\hat{\mu_{T+1|T}} = 0.0009, \hat{\sigma_{T+1|T}} = 0.0096$

\indent\indent $VaR(\log \text{returns})_{1\%, T + 1} = 0.0009 + 0.0096\times-3.27 = -0.0304$

\indent\indent $VaR_{1\%, T + 1} = \text{ Value in Euros } \times VaR(\log returns) = 10 000 000 \text{\euro} \times-0.0304 = -304332.17 \text{\euro}$\newline




Calculation for 90\% Confidence and $T+5$:

\indent\indent $\text{Quantile}_\text{std(5)}(0.1) = -1.46,\hat{\mu_{T+5|T}} = 0.001, \hat{\sigma_{T+5|T}} = 0.0103$

\indent\indent $VaR(\log \text{returns})_{10\%, T + 5} = 0.001 + 0.0103\times-1.46 = -0.0141$

\indent\indent $VaR_{10\%, T + 5} = \text{ Value in Euros } \times VaR(\log returns) = 10 000 000 \text{\euro} \times-0.0141 = -140990.25 \text{\euro}$\newline




Calculation for 95\% Confidence and $T+5$:

\indent\indent $\text{Quantile}_\text{std(5)}(0.05) = -1.99,\hat{\mu_{T+5|T}} = 0.001, \hat{\sigma_{T+5|T}} = 0.0103$

\indent\indent $VaR(\log \text{returns})_{5\%, T + 5} = 0.001 + 0.0103\times-1.99 = -0.0195$

\indent\indent $VaR_{5\%, T + 5} = \text{ Value in Euros } \times VaR(\log returns) = 10 000 000 \text{\euro} \times-0.0195 = -195229.37 \text{\euro}$\newline




Calculation for 99\% Confidence and $T+5$:

\indent\indent $\text{Quantile}_\text{std(5)}(0.01) = -3.27,\hat{\mu_{T+5|T}} = 0.001, \hat{\sigma_{T+5|T}} = 0.0103$

\indent\indent $VaR(\log \text{returns})_{1\%, T + 5} = 0.001 + 0.0103\times-3.27 = -0.0328$

\indent\indent $VaR_{1\%, T + 5} = \text{ Value in Euros } \times VaR(\log returns) = 10 000 000 \text{\euro} \times-0.0328 = -328409.98 \text{\euro}$\newline


Following the modeling assumptions of this subtask, the available stock data and the given sum invested in the Visa stock, the loss incurring over one day will be smaller than -130610.68$\text{\euro}$, -180885.55$\text{\euro}$  and -304332.17$\text{\euro}$  with a confidence level of 90\%, 95\%  and 99\%.

The loss after 5 days will be smaller than NA$\text{\euro}$, NA$\text{\euro}$  and NA$\text{\euro}$  with a confidence level of 90\%, 95\%  and 99\%.


\paragraph{Microsoft} Caclulations.\newline \indent 




Calculation for 90\% Confidence and $T+1$:

\indent\indent $\text{Quantile}_\text{std(7)}(0.1) = -1.41,\hat{\mu_{T+1|T}} = 0.0014, \hat{\sigma_{T+1|T}} = 0.0134$

\indent\indent $VaR(\log \text{returns})_{10\%, T + 1} = 0.0014 + 0.0134\times-1.41 = -0.0176$

\indent\indent $VaR_{10\%, T + 1} = \text{ Value in Euros } \times VaR(\log returns) = 10 000 000 \text{\euro} \times-0.0176 = -176045.17 \text{\euro}$\newline




Calculation for 95\% Confidence and $T+1$:

\indent\indent $\text{Quantile}_\text{std(7)}(0.05) = -1.88,\hat{\mu_{T+1|T}} = 0.0014, \hat{\sigma_{T+1|T}} = 0.0134$

\indent\indent $VaR(\log \text{returns})_{5\%, T + 1} = 0.0014 + 0.0134\times-1.88 = -0.024$

\indent\indent $VaR_{5\%, T + 1} = \text{ Value in Euros } \times VaR(\log returns) = 10 000 000 \text{\euro} \times-0.024 = -239879.38 \text{\euro}$\newline




Calculation for 99\% Confidence and $T+1$:

\indent\indent $\text{Quantile}_\text{std(7)}(0.01) = -2.97,\hat{\mu_{T+1|T}} = 0.0014, \hat{\sigma_{T+1|T}} = 0.0134$

\indent\indent $VaR(\log \text{returns})_{1\%, T + 1} = 0.0014 + 0.0134\times-2.97 = -0.0386$

\indent\indent $VaR_{1\%, T + 1} = \text{ Value in Euros } \times VaR(\log returns) = 10 000 000 \text{\euro} \times-0.0386 = -385621.58 \text{\euro}$\newline




Calculation for 90\% Confidence and $T+5$:

\indent\indent $\text{Quantile}_\text{std(7)}(0.1) = -1.41,\hat{\mu_{T+5|T}} = 0.0015, \hat{\sigma_{T+5|T}} = 0.0141$

\indent\indent $VaR(\log \text{returns})_{10\%, T + 5} = 0.0015 + 0.0141\times-1.41 = -0.0183$

\indent\indent $VaR_{10\%, T + 5} = \text{ Value in Euros } \times VaR(\log returns) = 10 000 000 \text{\euro} \times-0.0183 = -183242.66 \text{\euro}$\newline




Calculation for 95\% Confidence and $T+5$:

\indent\indent $\text{Quantile}_\text{std(7)}(0.05) = -1.88,\hat{\mu_{T+5|T}} = 0.0015, \hat{\sigma_{T+5|T}} = 0.0141$

\indent\indent $VaR(\log \text{returns})_{5\%, T + 5} = 0.0015 + 0.0141\times-1.88 = -0.025$

\indent\indent $VaR_{5\%, T + 5} = \text{ Value in Euros } \times VaR(\log returns) = 10 000 000 \text{\euro} \times-0.025 = -249994.08 \text{\euro}$\newline




Calculation for 99\% Confidence and $T+5$:

\indent\indent $\text{Quantile}_\text{std(7)}(0.01) = -2.97,\hat{\mu_{T+5|T}} = 0.0015, \hat{\sigma_{T+5|T}} = 0.0141$

\indent\indent $VaR(\log \text{returns})_{1\%, T + 5} = 0.0015 + 0.0141\times-2.97 = -0.0402$

\indent\indent $VaR_{1\%, T + 5} = \text{ Value in Euros } \times VaR(\log returns) = 10 000 000 \text{\euro} \times-0.0402 = -402396.65 \text{\euro}$\newline


Following the modeling assumptions of this subtask, the available stock data and the given sum invested in the Microsoft stock, the loss incurring over one day will be smaller than -176045.17$\text{\euro}$, -239879.38$\text{\euro}$  and -385621.58$\text{\euro}$  with a confidence level of 90\%, 95\%  and 99\%.

The loss after 5 days will be smaller than NA$\text{\euro}$, NA$\text{\euro}$  and NA$\text{\euro}$  with a confidence level of 90\%, 95\%  and 99\%.

The VaR obtained by the current models using t-stundent innovations are much larger than the ones obtained by the similar models using normaly distributed innovations. This is caused by the shift of likelohood towards the tails in t-student distributions. This can reflect an increased likelihood of extreme events like negative price shocks.
\subsubsection{(v) Other ARMA-GARCH specification}

\paragraph{Visa}
 An AR(0)-GARCH(2,3) model with normal innovations is used to obtain VaR values in this subtask. The model was chosen based on the problem statement and the Box-Jenkins-Methodology





Calculation for 90\% Confidence and $T+1$:

\indent\indent $\text{Quantile}_\text{std(5)}(0.1) = -1.46,\hat{\mu_{T+1|T}} = 0.001, \hat{\sigma_{T+1|T}} = 0.0096$

\indent\indent $VaR(\log \text{returns})_{10\%, T + 1} = 0.001 + 0.0096\times-1.46 = -0.0131$

\indent\indent $VaR_{10\%, T + 1} = \text{ Value in Euros } \times VaR(\log returns) = 10 000 000 \text{\euro} \times-0.0131 = -131014.75 \text{\euro}$\newline




Calculation for 95\% Confidence and $T+1$:

\indent\indent $\text{Quantile}_\text{std(5)}(0.05) = -1.98,\hat{\mu_{T+1|T}} = 0.001, \hat{\sigma_{T+1|T}} = 0.0096$

\indent\indent $VaR(\log \text{returns})_{5\%, T + 1} = 0.001 + 0.0096\times-1.98 = -0.0182$

\indent\indent $VaR_{5\%, T + 1} = \text{ Value in Euros } \times VaR(\log returns) = 10 000 000 \text{\euro} \times-0.0182 = -181531.61 \text{\euro}$\newline




Calculation for 99\% Confidence and $T+1$:

\indent\indent $\text{Quantile}_\text{std(5)}(0.01) = -3.27,\hat{\mu_{T+1|T}} = 0.001, \hat{\sigma_{T+1|T}} = 0.0096$

\indent\indent $VaR(\log \text{returns})_{1\%, T + 1} = 0.001 + 0.0096\times-3.27 = -0.0305$

\indent\indent $VaR_{1\%, T + 1} = \text{ Value in Euros } \times VaR(\log returns) = 10 000 000 \text{\euro} \times-0.0305 = -305375.38 \text{\euro}$\newline




Calculation for 90\% Confidence and $T+5$:

\indent\indent $\text{Quantile}_\text{std(5)}(0.1) = -1.46,\hat{\mu_{T+5|T}} = 0.001, \hat{\sigma_{T+5|T}} = 0.0103$

\indent\indent $VaR(\log \text{returns})_{10\%, T + 5} = 0.001 + 0.0103\times-1.46 = -0.014$

\indent\indent $VaR_{10\%, T + 5} = \text{ Value in Euros } \times VaR(\log returns) = 10 000 000 \text{\euro} \times-0.014 = -140149.55 \text{\euro}$\newline




Calculation for 95\% Confidence and $T+5$:

\indent\indent $\text{Quantile}_\text{std(5)}(0.05) = -1.98,\hat{\mu_{T+5|T}} = 0.001, \hat{\sigma_{T+5|T}} = 0.0103$

\indent\indent $VaR(\log \text{returns})_{5\%, T + 5} = 0.001 + 0.0103\times-1.98 = -0.0194$

\indent\indent $VaR_{5\%, T + 5} = \text{ Value in Euros } \times VaR(\log returns) = 10 000 000 \text{\euro} \times-0.0194 = -193941.91 \text{\euro}$\newline




Calculation for 99\% Confidence and $T+5$:

\indent\indent $\text{Quantile}_\text{std(5)}(0.01) = -3.27,\hat{\mu_{T+5|T}} = 0.001, \hat{\sigma_{T+5|T}} = 0.0103$

\indent\indent $VaR(\log \text{returns})_{1\%, T + 5} = 0.001 + 0.0103\times-3.27 = -0.0326$

\indent\indent $VaR_{1\%, T + 5} = \text{ Value in Euros } \times VaR(\log returns) = 10 000 000 \text{\euro} \times-0.0326 = -325815.67 \text{\euro}$\newline


Following the modeling assumptions of this subtask, the available stock data and the given sum invested in the Visa stock, the loss incurring over one day will be smaller than -131014.75$\text{\euro}$, -181531.61$\text{\euro}$  and -305375.38$\text{\euro}$  with a confidence level of 90\%, 95\%  and 99\%.

The loss after 5 days will be smaller than NA$\text{\euro}$, NA$\text{\euro}$  and NA$\text{\euro}$  with a confidence level of 90\%, 95\%  and 99\%.


\paragraph{Microsoft}
 An AR(0)-GARCH(2,2) model with normal innovations is used to obtain VaR values in this subtask. The model was chosen based on the problem statement and the Box-Jenkins-Methodology





Calculation for 90\% Confidence and $T+1$:

\indent\indent $\text{Quantile}_\text{std(7)}(0.1) = -1.41,\hat{\mu_{T+1|T}} = 0.0015, \hat{\sigma_{T+1|T}} = 0.0132$

\indent\indent $VaR(\log \text{returns})_{10\%, T + 1} = 0.0015 + 0.0132\times-1.41 = -0.0171$

\indent\indent $VaR_{10\%, T + 1} = \text{ Value in Euros } \times VaR(\log returns) = 10 000 000 \text{\euro} \times-0.0171 = -171297.95 \text{\euro}$\newline




Calculation for 95\% Confidence and $T+1$:

\indent\indent $\text{Quantile}_\text{std(7)}(0.05) = -1.89,\hat{\mu_{T+1|T}} = 0.0015, \hat{\sigma_{T+1|T}} = 0.0132$

\indent\indent $VaR(\log \text{returns})_{5\%, T + 1} = 0.0015 + 0.0132\times-1.89 = -0.0234$

\indent\indent $VaR_{5\%, T + 1} = \text{ Value in Euros } \times VaR(\log returns) = 10 000 000 \text{\euro} \times-0.0234 = -234064.61 \text{\euro}$\newline




Calculation for 99\% Confidence and $T+1$:

\indent\indent $\text{Quantile}_\text{std(7)}(0.01) = -2.97,\hat{\mu_{T+1|T}} = 0.0015, \hat{\sigma_{T+1|T}} = 0.0132$

\indent\indent $VaR(\log \text{returns})_{1\%, T + 1} = 0.0015 + 0.0132\times-2.97 = -0.0378$

\indent\indent $VaR_{1\%, T + 1} = \text{ Value in Euros } \times VaR(\log returns) = 10 000 000 \text{\euro} \times-0.0378 = -377535.41 \text{\euro}$\newline




Calculation for 90\% Confidence and $T+5$:

\indent\indent $\text{Quantile}_\text{std(7)}(0.1) = -1.41,\hat{\mu_{T+5|T}} = 0.0015, \hat{\sigma_{T+5|T}} = 0.0139$

\indent\indent $VaR(\log \text{returns})_{10\%, T + 5} = 0.0015 + 0.0139\times-1.41 = -0.0181$

\indent\indent $VaR_{10\%, T + 5} = \text{ Value in Euros } \times VaR(\log returns) = 10 000 000 \text{\euro} \times-0.0181 = -181407.55 \text{\euro}$\newline




Calculation for 95\% Confidence and $T+5$:

\indent\indent $\text{Quantile}_\text{std(7)}(0.05) = -1.89,\hat{\mu_{T+5|T}} = 0.0015, \hat{\sigma_{T+5|T}} = 0.0139$

\indent\indent $VaR(\log \text{returns})_{5\%, T + 5} = 0.0015 + 0.0139\times-1.89 = -0.0248$

\indent\indent $VaR_{5\%, T + 5} = \text{ Value in Euros } \times VaR(\log returns) = 10 000 000 \text{\euro} \times-0.0248 = -247582.23 \text{\euro}$\newline




Calculation for 99\% Confidence and $T+5$:

\indent\indent $\text{Quantile}_\text{std(7)}(0.01) = -2.97,\hat{\mu_{T+5|T}} = 0.0015, \hat{\sigma_{T+5|T}} = 0.0139$

\indent\indent $VaR(\log \text{returns})_{1\%, T + 5} = 0.0015 + 0.0139\times-2.97 = -0.0399$

\indent\indent $VaR_{1\%, T + 5} = \text{ Value in Euros } \times VaR(\log returns) = 10 000 000 \text{\euro} \times-0.0399 = -398843.04 \text{\euro}$\newline


Following the modeling assumptions of this subtask, the available stock data and the given sum invested in the Microsoft stock, the loss incurring over one day will be smaller than -171297.95$\text{\euro}$, -234064.61$\text{\euro}$  and -377535.41$\text{\euro}$  with a confidence level of 90\%, 95\%  and 99\%.

The loss after 5 days will be smaller than NA$\text{\euro}$, NA$\text{\euro}$  and NA$\text{\euro}$  with a confidence level of 90\%, 95\%  and 99\%.

