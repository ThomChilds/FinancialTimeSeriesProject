\section{Value at Risk (VaR) Analysis}
\label{sec:q2_value_at_risk_analysis}
\subsection{A - VaR calculations}

\subsubsection{Subtask(i)}

The only model that can be formed using just unconditional moments is the constant model $$R_t = \epsilon_t, \epsilon_t \overset{\text{w.n.}}{\sim} N(\mu, \sigma^2),$$
with $\hat \mu = \bar R_t, \hat{\sigma^2} = Var(R_t)$, for $t \in [1,T]$. 
Even for a random walk model both the standard deviation of an observation from the preceding one and a last observation would be needed. Both are not given.


To calculate the VaR of each stock, the historical mean and standard deviation (sd) will be estimated and the respective z value of a standard distribution will be scaled by the sd and shifted by the mean. 
Since the likelihood of the z value can be read off from the normal distribution, the z value corresponding to the likelihood of interest can be chosen.


The scaled and shifted z value corresponds to a maximal log return under the given probability. The typically small log returns of stocks are proportional to percent changes in the stock value. Tue to this, maximal the monetary value of currency that is at risk can be calculated by multiplying the result of the previous calculation with the value of bought stock.


\paragraph{MSFT Stock}
